\documentclass[UTF8]{ctexart}
\usepackage{geometry, CJKutf8}
\geometry{margin=1.5cm, vmargin={0pt,1cm}}
\setlength{\topmargin}{-1cm}
\setlength{\paperheight}{29.7cm}
\setlength{\textheight}{25.3cm}

% useful packages.
\usepackage{amsfonts}
\usepackage{amsmath}
\usepackage{amssymb}
\usepackage{amsthm}
\usepackage{enumerate}
\usepackage{graphicx}
\usepackage{multicol}
\usepackage{fancyhdr}
\usepackage{layout}
\usepackage{listings}
\usepackage{float, caption}

\lstset{
    basicstyle=\ttfamily, basewidth=0.5em
}

% some common command
\newcommand{\dif}{\mathrm{d}}
\newcommand{\avg}[1]{\left\langle #1 \right\rangle}
\newcommand{\difFrac}[2]{\frac{\dif #1}{\dif #2}}
\newcommand{\pdfFrac}[2]{\frac{\partial #1}{\partial #2}}
\newcommand{\OFL}{\mathrm{OFL}}
\newcommand{\UFL}{\mathrm{UFL}}
\newcommand{\fl}{\mathrm{fl}}
\newcommand{\op}{\odot}
\newcommand{\Eabs}{E_{\mathrm{abs}}}
\newcommand{\Erel}{E_{\mathrm{rel}}}

\begin{document}

\pagestyle{fancy}
\fancyhead{}
\lhead{姓名: 邓东宁}
\chead{数据结构与算法作业四}
\rhead{学号: 3230101177}
\cfoot{Oct.18th, 2024}

\section{测试程序的设计思路}

概括的测试历程是“初始化->增删改查”(其中“改”和“查”是我补充的操作)

\subsection{初始化部分}
\begin{itemize}
  \item 用 \texttt{lst1} 测试初始化列表构造
  \item 用 \texttt{lst2} 测试拷贝构造
  \item 先初始化一个空的 \texttt{lst3},再藉助赋值运算符用 \texttt{lst2} 对其进行初始化
  \item 藉助移动构造函数用 \texttt{lst3} 初始化 \texttt{lst4},并确认过后 \texttt{lst3} 已无数据
\end{itemize}

\subsection{增删改查部分}
(以下所有测试基于 \texttt{lst4})
\begin{itemize}
  \item 测试用 \texttt{push\_front()} 和 \texttt{push\_back()} 分别在头部和尾部插入数据
  \item 测试用 \texttt{pop\_front()} 和 \texttt{pop\_back()} 分别移除头部和尾部的数据
  \item 藉助自增符号移动迭代器指针,以测试用 \texttt{insert()} 在非头尾处插入数据
  \item 测试用 \texttt{erase()} 移除刚才插入的数据
\end{itemize}
\begin{itemize}
  \item 藉助自减符号逆序输出 \texttt{lst4},以测试本次作业内容之一自减操作之有效性
  \item 测试用 \texttt{update()} 更新当前节点的数据
  \item 测试用 \texttt{find()} 分别查找一个已知存在、一个已知不存在的数据
\end{itemize}
\begin{itemize}
  \item 测试用 \texttt{clear()} 清空各节点
\end{itemize}

\section{测试的结果}

\subsection{初始化测试}
所有初始化测试结果正常。

\subsection{增删改查测试}
增删查操作均符合预期,未发现异常。

\subsection{内存管理}
我尚未使用 \texttt{valgrind} 测试内存泄漏的潜在性。烦请助教检验

\end{document}
