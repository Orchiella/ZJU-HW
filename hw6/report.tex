\documentclass[UTF8]{ctexart}
\usepackage{geometry, CJKutf8}
\geometry{margin=1.5cm, vmargin={0pt,1cm}}
\setlength{\topmargin}{-1cm}
\setlength{\paperheight}{29.7cm}
\setlength{\textheight}{25.3cm}

% useful packages.
\usepackage{amsfonts}
\usepackage{amsmath}
\usepackage{amssymb}
\usepackage{amsthm}
\usepackage{enumerate}
\usepackage{graphicx}
\usepackage{multicol}
\usepackage{fancyhdr}
\usepackage{layout}
\usepackage{listings}
\usepackage{float, caption}

\lstset{
    basicstyle=\ttfamily, basewidth=0.5em
}

% some common command
\newcommand{\dif}{\mathrm{d}}
\newcommand{\avg}[1]{\left\langle #1 \right\rangle}
\newcommand{\difFrac}[2]{\frac{\dif #1}{\dif #2}}
\newcommand{\pdfFrac}[2]{\frac{\partial #1}{\partial #2}}
\newcommand{\OFL}{\mathrm{OFL}}
\newcommand{\UFL}{\mathrm{UFL}}
\newcommand{\fl}{\mathrm{fl}}
\newcommand{\op}{\odot}
\newcommand{\Eabs}{E_{\mathrm{abs}}}
\newcommand{\Erel}{E_{\mathrm{rel}}}

\begin{document}

\pagestyle{fancy}
\fancyhead{}
\lhead{姓名: 邓东宁}
\chead{数据结构与算法作业六}
\rhead{学号: 3230101177}
\cfoot{Nov.10th, 2024}

\section{remove函数的设计思路}
前半部分(查找与删除)代码基本和上次作业的一致,我进行了写法上的优化,但分类讨论情况与上次一致。后半部分(自平衡操作)从根节点遍历到删除节点原本的位置构造一条路径,接着沿着路径从下至上依次执行平衡操作。\\
至于平衡操作,我构造了多个辅助函数,包括获取高度、更新高度、左旋转、右旋转。在平衡函数中,会根据实际情况,藉助左旋或右旋来完成平衡操作。

\section{测试的结果}
最终结果符合预期,未发现报错与超时的情况

\section{遇到的困难}
1.在最开始,我想推翻之前的BST remove代码,以尝试用递归的方法来删除节点,因为这样能做到在堆栈的顶层删除后,依次回溯执行平衡操作,遍历次数会更少。但当遇到测试程序$N=300000$的情形时,显然超时了,并提示了某个非0的exit code,经百度得知这个code对应栈溢出的情形,毕竟递归深度确实过大,于是我只能放弃了递归法。\\
2.在构造上述提到的“路径”时,我不知道有什么合适的容器能表示下任意大小的集合(毕竟测试程序里的N还可以任意大),于是求助了AI,才知道原来有一个叫vector的动态数组类型,看来是我见识还不够,以后得多接触点才行。


\end{document}
