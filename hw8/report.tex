\documentclass[UTF8]{ctexart}
\usepackage{geometry, CJKutf8}
\geometry{margin=1.5cm, vmargin={0pt,1cm}}
\setlength{\topmargin}{-1cm}
\setlength{\paperheight}{29.7cm}
\setlength{\textheight}{25.3cm}

% useful packages.
\usepackage{amsfonts}
\usepackage{amsmath}
\usepackage{amssymb}
\usepackage{amsthm}
\usepackage{enumerate}
\usepackage{graphicx}
\usepackage{multicol}
\usepackage{fancyhdr}
\usepackage{layout}
\usepackage{listings}
\usepackage{float, caption}

\lstset{
    basicstyle=\ttfamily, basewidth=0.5em
}

% some common command
\newcommand{\dif}{\mathrm{d}}
\newcommand{\avg}[1]{\left\langle #1 \right\rangle}
\newcommand{\difFrac}[2]{\frac{\dif #1}{\dif #2}}
\newcommand{\pdfFrac}[2]{\frac{\partial #1}{\partial #2}}
\newcommand{\OFL}{\mathrm{OFL}}
\newcommand{\UFL}{\mathrm{UFL}}
\newcommand{\fl}{\mathrm{fl}}
\newcommand{\op}{\odot}
\newcommand{\Eabs}{E_{\mathrm{abs}}}
\newcommand{\Erel}{E_{\mathrm{rel}}}

\begin{document}

\pagestyle{fancy}
\fancyhead{}
\lhead{姓名: 邓东宁}
\chead{数据结构与算法作业七}
\rhead{学号: 3230101177}
\cfoot{Nov.29th, 2024}

\section{mySort()函数的设计思路}
先把传入的容器用\textbf{make$\_$heap()}建堆,然后利用\textbf{pop$\_$heap()}一边 交换最大元和末尾元 一边 恢复堆结构 的性质,设计一个遍历,依次对由前n个节点构成的子堆$(n=size,size-1,...,2)$执行\textbf{pop$\_$heap()}函数,以逐步把节点元素从大到小排到末尾,最终得到一个升序容器

\section{测试的流程}
构造了\textbf{getRandomVec(),getOrderedVec(),getRepeatedVec()}函数分别生成要求的测试用例。\\
接着定义了用例测试函数\textbf{testCase()},分别执行并记录两种sort函数历时

\section{测试的结果}
\begin{table}[ht]
\centering
\begin{tabular}{|c|c|c|}  % 表示三列,使用竖线分隔
\hline  % 水平线
& mySort()& std::sort$\_$heap()\\  % 第一行内容
\hline
random& 0.585373s& 0.517296s\\  % 第二行内容
\hline
ordered& 0.424473s& 0.37941s\\  % 第三行内容
\hline
 reverse& 0.454048s&0.38589s\\
\hline
 repetitive& 0.420623s&0.379411s\\
\hline
\end{tabular}
\caption{mySort() vs std::sort$\_$heap()}
\end{table}
实际上我进行了多次测试,以上结果为其中的随机一次,但也注意到,在每次测试的各个实例中,\textbf{前者效率总是稍逊于后者}

\section{时间复杂度分析}
先分析我定义的mySort()函数\\
第一步是建堆:\\
假设规模为$N$,对应堆层数$h=[log_{2}N](+1)$\\
调节至最大堆的过程如下\\
\begin{itemize}
  \item 从最后一个叶节点的父节点开始往回遍历,即$i=n/2-1$ to $0$,依次使用\textbf{下沉操作}
  \item 下沉操作即把当前节点与\textbf{左右子节点中最大者}调换,再对调换下去后的节点\textbf{递归}进行下沉操作
  \item 对于第$r$层的节点,将至多调用$(h-r)$次下沉操作,而第$r$层至多有$2^r$个节点会调用下沉操作(实际上除了遍历起点的那一层,上面的层都全部会调用一次),得到该层至多调用$(h-r)*2^r$次
  \item 对每一层调用下沉操作的次数进行求和,得到表达式$T_1=\sum_{r=1}^{h-1}(h-r)*2^r$,用高中时最爱的错位相减法化简表达式得到$T=2^{h+1}-2-2h$,代入$h$表达式,得到$T_1=O(N-logN)=O(N)$
\end{itemize}
下面分析第二步,即反复把子堆的最大元移到末尾来实现排序:
\begin{itemize}
\item 每一次把最大元换到末尾后,会对换上来的新节点执行\textbf{下沉操作},这个过程逐层往下,一层一次,则下沉次数至多为堆高$h$(经过分析,下沉次数会随着子堆层数减少而减小,为子堆当前的层数,即很容易就小于$h$,但不影响最终估计,毕竟指数这个东西太强了)
\item 这样操作要遍历$N$次,故下沉总次数$T_2=O(Nh)=O(NlogN)$
\end{itemize}
现在我们得到了总的时间复杂度$T=T_1+T_2=O(N)+O(NlogN)=O(NlogN)$\\
\\

\section{效率差异的原因}
经过查阅资料,\textbf{std::sort$\_$heap()}函数的思路其实和我的mySort()是一致的,但为什么前者稍快呢?其实原理很简单,因为\textbf{std::sort$\_$heap()}需要传入一个\textbf{已经满足最大堆性质的集合},在测试程序中因为只要求测试封装好的\textbf{std::sort$\_$heap()}函数,所以我在调用\textbf{std::make$\_$heap()}后才开始计时,这时候相当于赢在起跑线上了,\textbf{少比\textbf{mySort()}执行一个时间复杂度为$O(N)$的建堆工作},但由于这个时间复杂度相对于主项$O(NlogN)$阶数更小,所以\textbf{std::sort$\_$heap()}虽然更快,但优势不明显。\\\\在我后续的实验中,我把\textbf{std::make$\_$heap()}加入计时区域后,发现两种算法运行得时间其实就平分秋色了,不再显现出孰优孰劣,与我的推断切合。


\section{遇到的困难}
经查阅资料,发现建堆和平衡堆的关键操作是\textbf{下沉操作},学习理解这个过程也花了点时间。


\end{document}
